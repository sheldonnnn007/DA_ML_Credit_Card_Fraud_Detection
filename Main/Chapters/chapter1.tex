%\chapter{Introduction}
%\label{chapter1}
%
%\section{Context}
%\lipsum[1-1] \cite{parikh1980adaptive}
%
%\section{Context2}
\chapter{Introduction} % This starts a new chapter

%\section{Project Structure} % This starts a new section
% Here you can put the contents of the section


\section{Problem Statement}
% Here you can put the contents of the section
Credit card transaction fraud is when fraudsters steal consumer credit information through various means and then impersonate the cardholder to make purchases. It is a very common financial issue. When fraud occurs, the cardholder's financial security is threatened, and merchants risk losing merchandise and being punished. In this age of rapid information technology development, cash transactions are becoming less and less frequent, and transaction fraud is constantly occurring, posing a significant threat to people's financial security.\\
\vspace{12pt}
When a transaction takes place, various attributes are recorded simultaneously (such as transaction time, location, credit card number, amount, etc.) at the service provider's end [CreditCardFraudDetection4]. This data can be used to check for fraudulent transactions and even interrupt suspicious transactions before they are completed and contact the cardholder.\\
\vspace{12pt}
The main problem is the human inability to instantly recognise fraudulent transactions. As a result, the implementation of machine learning algorithms become important for the analysis of transactional data and detecting such fraudulent activities. In this context, a range of algorithmic models is utilised. And these models' performance is then systematically evaluated through comparative analysis.

\section{Aim and Objectives}

\subsection{Aim}
% Here you can put the contents of the subsection
This project aims to:
\begin{enumerate}
  \item Identify the variables that have the highest predictive power for fraudulence, meaning the variables that have the most significant impact on the outcome.
  \item Compare various models and determine which model has the highest accuracy, as well as understanding the reasons for its superior performance.
\end{enumerate}
%\newpage
\subsection{Objectives}
The objectives of this project are to:
\begin{enumerate}
  \item Perform data pre-processing, including using Synthetic Minority Oversampling Technique (SMOTE) on credit card transactions data, and conduct a visualisation analysis and Z-test to detect any potential patterns and determine the most impactful variables.
  \item Utilise both supervised and unsupervised machine learning algorithms for detecting transactional fraud, including logistic regression, random forest, Support Vector Machine (SVM), Isolation Forest, and Local Outlier Factor. Additional algorithms, such as Naive Bayes Classifier and Autoencoder, may be utilised if time permits.
  \item Conduct several performance evaluation metrics, including accuracy, precision, recall, and ROC curve analysis, to assess the effectiveness of the models. This will be followed by a detailed analysis of the underlying factors contributing to the superior performance of the selected models. If a better approach is discovered during project implementation, it will be adopted.
  \item Present a summary of the key points related to Anomaly Detection, highlight the most important takeaways from the analysis, and propose improvement measures.
\end{enumerate}


%\subsection{Deliverables}
% Here you can put the contents of the subsubsection

\section{Risk Assessment}
% Here you can put the contents of the section
%\documentclass{article}
\begin{table}[htbp]
\centering
\small
\setlength\tabcolsep{2pt}
\begin{tabular}{|c|c|c|c|}
\hline
\textbf{Risk} & \textbf{Likelihood} & \textbf{Impact} & \textbf{Mitigation plan} \\
\hline
No original data provided because confidence  & Cell 2 & Cell 3 & Cell 4 \\
\hline
Software bugs & Cell 6 & Cell 7 & Cell 8 \\
\hline
No enough time to finish full project & Medium & High & Cell 12 \\
\hline
File Missing & Low & High & Use the Internet hosting service to keep files safe \\
\hline
\end{tabular}
\caption{Table Title}
\label{tab:table-label}
\end{table}

%\end{document}


%\section{Planning and Project Management}
%% Here you can put the contents of the section
%
%\subsection{Proposed Plan}
%% Here you can put the contents of the subsection

